\documentclass{UNSAM}
\usepackage[utf8]{inputenc}

\addbibresource{bibliografia.bib}

\makeglossaries
%Subsistemas
\newacronym{obc}{OBC}{On Board Computer}
\newacronym{eps}{EPS}{Electric Power System}
\newacronym{gnc}{GNC}{Guidance Navigation and Control}
\newacronym{ttyc}{TT\&C}{Tracking Telemetry and Communication}
\newacronym{tcs}{TCS}{Thermal Control System}

\newacronym{leo}{LEO}{Low Earth Orbit}
\newacronym{inspa}{InSPA}{In-Space Production Applications}


%revisiones
\newacronym{pdr}{PDR}{Preliminary desing review}
\newacronym{mcr}{MCR}{Mission Concept Review}
\newacronym{cdr}{CDR}{Revisión Crítica de Diseño}

\newacronym{iise}{IISE}{Introducción a la Ingenieria de Sistemas Espaciales}
\newacronym{unsam}{UNSAM}{Universidad Nacional de San Martin}

\newacronym{pdu}{PDU}{Proceso Unificado de Desarrollo}
\newacronym{csv}{CSV}{Valores separados por coma}
\newacronym{adc}{ADC}{Conversor Analógico a Digital}

\input{glosario.tex}

\begin{document}
\includepdf[pages=1]{portada.pdf}
\titulo{U-SPACE-FACT}
\subtitulo{Revisión de Concepto de Misión}
\materia{IISE}
\fecha{2025}
\docn{2}

\setcounter{page}{1}
\tableofcontents
\listoffigures
\listoftables

\printglossary[type=\acronymtype]


\chapter{Planteamiento de la Misión}

\section{Necesidad}
Existen manufacturas que se ven muy beneficiadas de condiciones de microgravedad fácilmente obtenibles en órbita.
Por ejemplo, la cristalografía bajo las condiciones de gravedad propias de la superficie terrestre tiene estos de problemas:
\begin{itemize}
    \item \textbf{Sedimentación:} durante el crecimiento de cristales, la gravedad genera el desplazamiento de 
      moléculas hacia abajo, lo que altera la uniformidad del cristal.

    \item \textbf{Convección térmica:} la gravedad induce corrientes de convección en el fluido, provocadas por diferencias 
      de temperatura, que afectan la distribución homogénea de las moléculas.

    \item \textbf{Cristalografía de Rayos X:} Los cristales con defectos dan patrones de difracción menos precisos.
\end{itemize}
Esto puede generar defectos estructurales, zonas de impurezas o crecimientos irregulares.
\\
¿Por qué importa esto?
\\
En campos como la farmacéutica, nanotecnología o materiales avanzados, una pequeña variación en la estructura 
cristalina puede cambiar completamente el comportamiento del producto.


\section{Contexto}
La NASA ha seleccionado nuevas propuestas dentro del programa \acrfull{inspa}, cuyo objetivo principal es avanzar en 
el desarrollo de capacidades industriales en órbita baja terrestre (\acrshort{leo}). Estas iniciativas buscan 
transformar la microgravedad en una herramienta productiva, permitiendo la fabricación de materiales y productos con 
prestaciones superiores a las obtenidas en la Tierra.
\\
Las propuestas seleccionadas se centran en dos áreas de alto impacto:
\begin{itemize}
\item Producción de fibras ópticas avanzadas, especialmente del tipo ZBLAN. En ausencia de gravedad, el proceso de 
  fabricación reduce significativamente defectos como cristalización interna e impurezas, lo que permite obtener fibras 
    con mayor calidad y eficiencia para aplicaciones en telecomunicaciones y transmisión de datos.

\item Aplicaciones biomédicas basadas en células madre, orientadas al desarrollo de tejidos y modelos celulares 
  tridimensionales. La microgravedad favorece la formación de estructuras más complejas y realistas, con potencial uso 
    en investigación médica, tratamientos y desarrollo farmacéutico.
\end{itemize}
La NASA destaca que estas actividades representan un avance concreto hacia la creación de una economía en \acrshort{leo}.
Para que esta economía sea sostenible, resulta esencial desarrollar tecnologías específicas de manufactura orbital 
que sean escalables, repetibles y con niveles crecientes de automatización. Asimismo, se resalta la necesidad de 
reducir la dependencia de operaciones tripuladas, promoviendo sistemas capaces de producir y retornar productos a la 
Tierra con mínima intervención humana.
\\
\href{https://www.nasa.gov/humans-in-space/commercial-space/nasa-selects-proposals-in-space-development-optical-fibers-stem-cells-enable-low-earth-orbit-leo-economy}{NASA}



\section{Metas}
\begin{itemize}
  \item Brindar un servicio que habilite manufacturas en órbita \acrshort{leo}.
  \item Competir de igual a igual con grandes empresas.
  \item Establecer estándares para fabricación en masa y comercializar patentes.
  \item Minimizar riesgos en todo el ciclo de vida del producto.
\end{itemize}

\section{Objetivos}
\begin{itemize}
  \item Transportar, operar y dar soporte a una fábrica en órbita \acrshort{leo}.
  \item Recuperar los productos fabricados.
  \item Insertarse en el mercado de fabricación en órbita.
  \item Desarrollar métodos y tecnologías eficientes, rentables y reutilizables.
  \item Colaborar con universidades y lograr acuerdos de financiamiento futuros.
\end{itemize}

\section{Hipótesis}

\subsection{rentabilidad}
La baja del costo de lanzamiento va a generar que sea rentable.

\subsection{Impacto sobre la cristalización:}

La cristalización macromolecular busca formar cristales altamente ordenados capaces de difractar rayos X o neutrones 
con buena resolución. La calidad del cristal depende en gran medida de los procesos de transporte y crecimiento dentro 
de la solución, y estos pueden cambiar cuando se reduce la aceleración, como en microgravedad.

En microgravedad disminuyen la convección y la sedimentación, por lo que el transporte de macromoléculas hacia el 
cristal queda dominado por difusión. Esto puede reducir la incorporación de agregados grandes e impurezas, aunque los 
resultados dependen de la concentración y de la pureza inicial de la muestra, que se vuelve un factor crítico para 
obtener beneficios reales.

La nucleación inicial no cambia de forma directa, ya que está gobernada por fuerzas intermoleculares que no dependen 
de la gravedad. Sin embargo, la microgravedad puede reducir la nucleación secundaria al eliminar flujos de fluido que, 
en la Tierra, desprenden fragmentos cristalinos que actúan como nuevos núcleos.

Durante el crecimiento, la ausencia de convección favorece la formación de una zona de agotamiento estable alrededor 
del cristal, permitiendo un crecimiento más lento y ordenado. Esto facilita la expulsión de unidades mal incorporadas, 
mejorando el orden interno.

Los estudios muestran que los cristales crecidos en microgravedad suelen presentar menor mosaicidad, dominios más 
homogéneos y mayor resolución de difracción, lo que se traduce en datos estructurales de mayor calidad. La importancia 
de estos efectos se vuelve especialmente evidente cuando se consideran futuras aplicaciones industriales y biomédicas 
que dependen de estructuras cristalinas precisas \cite{snell2021microgravity}.

\begin{figure}[H]
    \centering
    \includegraphics[scale=1]{Imagenes/Figura que no entiendo.png}
    \caption{Estudios experimentales de cristales crecidos en microgravedad muestran dominios más uniformes y una mosaicidad
    significativamente menor, como indican los perfiles de rocking y los análisis estadísticos de reflexiones. Esto
    produce una mayor intensidad de las reflexiones y, en consecuencia, una mejor relación señal–ruido. Los resultados
    observados en cristales de lisozima se confirmaron posteriormente en cristales de insulina mediante una metodología
    mejorada .}
    \label{fig:hipotesis}
\end{figure}

\begin{figure}[H]
    \centering
    \includegraphics[scale=0.25]{Imagenes/Diferencia.jpg}
    \caption{Microscopio de escaneo por electrones: $\mu$g(izquierda) y $1$g(derecha)\cite{torres2016experimental}}
    \label{fig:Diferencia de cristalisacion}
\end{figure}

\section{Misión}
Transportar y colocar en órbita \acrshort{leo} un satélite equipado con una fábrica, monitorear dicha fábrica durante el
tiempo requerido para que produzca la cantidad especificada de producto en microgravedad, almacenar los productos
obtenidos bajo condiciones controladas y finalmente reingresar el satélite junto con los productos de forma segura a 
la Tierra para su recuperación y uso.


\section{Restricciones}
\begin{itemize}
    \item Fabricación de algunas manufacturas.
    \item Limitaciones de espacio en cápsula para albergar fábricas de mayor tamaño.
    \item Presupuesto para la misión completa, que sea rentable.
    \item Mercado reducido y específico.
    \item Importación de materiales y tecnología.
\end{itemize}

\section{Autoridades}
\begin{itemize}
  \item Lanzamiento:  el transportador – Responsable:Space X
  Durante toda la fase de lanzamiento, incluidos integración en el vehículo, validaciones previas, carga en la cofia, cronología 
  de despegue y ascenso, la responsabilidad operativa y legal recae completamente en el proveedor de lanzamiento.
  Este actor garantiza la seguridad del vehículo, el cumplimiento normativo y la integridad del payload hasta la separación en órbita.

\item Operación de la cápsula en órbita – Responsable: X \acrshort{unsam}
  Una vez que la cápsula se separa del lanzador e ingresa en régimen orbital, el control y la autoridad operacional pasan al Grupo X.

  \item Operación de la fábrica interna – Responsable: el cliente
  La carga útil (la “fábrica orbital”) es operada exclusivamente por el cliente.Grupo X provee los medios (energía, 
  telemetría, ambiente controlado), pero no interviene en las decisiones operativas internas del payload.

\item Reingreso, recuperación y custodia – Responsable: Grupo X (\acrshort{unsam})
  Al finalizar la campaña, Grupo X ejecuta la maniobra de reingreso, gestiona la trayectoria balística y supervisa la
  apertura de sistemas de frenado.
  Tras el aterrizaje, Grupo X asume la logística de recuperación de la capsula, aseguramiento del sitio de aterrizaje,
  verificación de integridad y traslado seguro de la cápsula hasta la base de operaciones.
  El cliente toma posesión del producto una vez finalizados los procedimientos de entrega controlada.

\end{itemize}

\section{ConOps (Concepto de Operación)}
El proceso se divide en 8 fases principales:

\begin{enumerate}
  \item \textbf{Lanzamiento y separación}: El satélite U-SPACE-FACT es lanzado y colocado en \acrshort{leo}. Condición 
    de entrada: separación confirmada y telemetría nominal de bus (\acrshort{eps}, \acrshort{obc}, \acrshort{gnc}).
  \item \textbf{Fase de comisionado (\acrshort{leo} early ops)}: Inicialización de subsistemas y verificación de salud 
    (health check) de la fábrica (cámaras, controladores Peltier, actuadores, cartuchos).
    \item \textbf{Inicio de operaciones de fabricación}: Carga de la receta, puesta en marcha del primer ciclo, 
      monitorización en tiempo real y test de funcionamiento inmediato tras el primer ciclo.
    \item \textbf{Campaña de producción}: Operar por $N$ días o hasta alcanzar la masa/número de muestras requerido. 
      Registro continuo ALCOA+ y inspecciones intermedias on-orbit.
    \item \textbf{Verificación de cantidad y calidad}: Ejecutar rutina de verificación final on-orbit. Decisión de 
      aceptación para reentrada si calidad $\geq$ criterios mínimos, sino se decide retrabajo/descarte/retención.
    \item \textbf{Preparación para reingreso}: Transferencia y sellado de muestras, acondicionamiento, y configuración de 
      la cápsula para desacoplamiento y maniobra de reentrada.
    \item \textbf{Maniobra de reingreso y recuperación}: Ejecución de desorbitación y reingreso seguro. Condición de éxito: 
      cápsula recuperada y muestras en condición aceptable.
    \item \textbf{Post-recuperación y cierre de campaña}: Transporte a instalaciones de análisis en Tierra, análisis full-
      panel on-ground y reporte de comparabilidad.
\end{enumerate}

\begin{figure}[H]
    \centering
    \includegraphics[scale=0.62]{Imagenes/ConOps.png}
    \caption{Concepto de Operación 1}
    \label{fig:Concepto de Operación}
\end{figure}

\newpage
\chapter{Desglose del Sistema}

\section{Diagrama de bloques funcional del sistema}
\begin{figure}[H]
    \centering
    \includegraphics[scale=0.7]{Imagenes/1. Concept Level/Structure/Feature_Context.png}
    \caption{Concepto funcional}
    \label{fig:Concepto funcional}
\end{figure}

\section{Diagrama de subsistemas}
Subsistemas:

\begin{itemize}
  \item \acrfull{obc}: coordina y ejecuta el control de misión, administra datos y gestiona la operación del sistema.
  \item \acrfull{eps}: Genera, almacena y distribuye la energía eléctrica necesaria para todos los subsistemas.
  \item \acrfull{gnc}: Determina la actitud y posición del vehículo y ejecuta maniobras para mantener la orientación y 
    trayectoria requeridas.
  \item \acrfull{ttyc}: Mantiene el enlace de comunicaciones con tierra, transmitiendo telemetría y recibiendo comandos.
  \item \acrfull{tcs}: Regula la temperatura del vehículo y del payload para mantenerlos dentro de rangos operativos seguros.
  \item Payload: Realiza el proceso de fabricación en microgravedad y produce el output utilizable por el cliente.
  \item Reentrada: Gestiona el retorno atmosférico y asegura la desaceleración, protección térmica y recuperación segura.
  \item Estructura: Provee soporte mecánico, protege los componentes internos y transmite las cargas del lanzamiento, 
    operación y reentrada.
\end{itemize}

La figura \ref{fig:subsistemas} presenta un diagrama funcional de subsistemas, centrado en las interacciones operativas entre los 
elementos principales de la misión. El objetivo no es mostrar la totalidad del equipamiento embarcado, sino 
representar cómo los subsistemas intercambian información, control y recursos durante el funcionamiento nominal.

Ciertos elementos físicos —como la estructura, el termo–óptico o recubrimientos pasivos— no se incluyen aquí, ya que, 
si bien son indispensables desde el punto de vista ingenieril, no intervienen directamente en los flujos funcionales 
que este esquema busca destacar. Su presencia se asume como parte del soporte básico de la plataforma.

\begin{figure}[H]
    \centering
    \includegraphics[scale=0.7]{Imagenes/2. Logical Level/Structure/Logical_Boundary.png}
    \caption{Diagrama de subsistemas}
    \label{fig:Diagrama de subsistemas}
    \label{fig:subsistemas}
\end{figure}

\section{Diagrama de N2 (Interfaces)}

\begin{figure}[H]
    \centering
    \includegraphics[width=\textwidth]{Imagenes/n2.png}
    \caption{Diagrama N2}
    \label{fig:Diagrama N2}
\end{figure}

\chapter{Ambiente de Misión}

\section{Fase de Lanzamiento}

\begin{table}[h!]
\centering

\begin{tabularx}{\textwidth}{|p{4cm}|X|}
\hline
\rowcolor{gray!20}
\textbf{Factor Ambiental} & \textbf{Rango / Descripcion} \\
\hline

{\raggedright \textbf{Vibraciones Acusticas}} &
Ruidos de banda ancha de \textbf{140 dB a 150 dB} generados por los motores del cohete. \\
\hline

{\raggedright \textbf{Vibraciones Aleatorias \\ y Sinusoidales}} &
Movimiento de alta frecuencia durante el encendido de los motores y el vuelo. \\
\hline

{\raggedright \textbf{Aceleraciones (Fuerzas G)}} &
Pico de aceleracion vertical (típicamente\textbf{3G a 9G}), especialmente en la etapa final del encendido. \\
\hline

{\raggedright \textbf{Choque de Separacion}} &
Pulso mecanico de muy alta frecuencia (hasta \textbf{10{,}000 Hz}) causado por la liberacion de tuercas explosivas. \\
\hline

{\raggedright \textbf{Presion Atmosferica}} &
Transicion de \textbf{1 atm (nivel del mar)} a \textbf{casi vacio} en orbita. \\
\hline

\end{tabularx}
\caption{Ambiente durante fase de lanzamiento}
\end{table}

\section{Fase de Órbita}
El ambiente de operación principal, caracterizado por el vacío, temperaturas extremas y radiación.

\begin{table}[h!]
\centering
\renewcommand{\arraystretch}{1.4}

\begin{tabularx}{\textwidth}{|>{\raggedright\arraybackslash}p{4cm}|X|}
\hline
\rowcolor{gray!20}
\textbf{Factor Ambiental} & \textbf{Rango / Descripción} \\
\hline

\textbf{Temperatura Extrema} &
-100°C a +100°C \\
\hline

\textbf{Vacío Extremo} &
Rangos de presión extremos (típicamente\textbf{$10^-3Pa$ a $10^-6Pa$})\\
\hline

\textbf{Microgravedad} &
Aceleración residual muy baja (típicamente\textbf{$10^-6G$ a $10^-4G$})\\
\hline

\textbf{Radiación Espacial} &
\textbf{Protones de alta energía} (especialmente en la Anomalía del Atlántico Sur, SAA) (típicamente\textbf{$100rad$ a $2000rad$}). \\
\hline

\textbf{Basura Espacial (Debris)} &
Partículas entre 7km/s a 15km/s. \\
\hline

\end{tabularx}
\caption{Ambiente durante fase de órbita}
\end{table}

\section{Fase de Reingreso}
Implica el soporte de las condiciones más extremas de calor y presión.

\begin{table}[h!]
\centering
\renewcommand{\arraystretch}{1.4}

\begin{tabularx}{\textwidth}{|>{\raggedright\arraybackslash}p{4cm}|X|}
\hline
\rowcolor{gray!20}
\textbf{Factor Ambiental} & \textbf{Rango / Descripción} \\
\hline

\textbf{Carga Térmica Extrema} &
Temperaturas superficiales de \textbf{1.000°C a 5.000°C}. \\
\hline

\textbf{Altas Fuerzas G} &
  Desaceleración máxima por el frenado atmosférico (típicamente \textbf{$10G$ a $20G$)}. \\
\hline

\textbf{Presión Dinámica Máxima (Max Q)} &
  Presión aerodinámica intensa sobre la superficie (típicamente \textbf{$10kPa$ a $60kPa$)}. \\
\hline

\textbf{Plasma Ionizado} &
  Vaina de gas ionizado que envuelve el objeto (típicamente \textbf{$1017 electrones/m3$ a $1019 electrones/m3$)}. \\
\hline

\end{tabularx}
\caption{Ambiente durante fase de reingreso}
\end{table}

\chapter{Planificación del Proyecto}

\section{Work Breakdown Structure (WBS)}
\begin{enumerate}
    \item \textbf{U-SPACE-FACT}
    \begin{enumerate}
        \item \textbf{Gestión del Proyecto}
        \item \textbf{Ingeniería de Sistemas}
        \item \textbf{Plataforma Satelital}
        \begin{enumerate}
            \item Estructura
            \item \acrshort{eps}
            \item \acrshort{gnc}
            \item \acrshort{obc}
        \end{enumerate}
        \item \textbf{Payload – Laboratorio}
        \begin{enumerate}
            \item Control Térmico
            \item Cámaras
            \item Algoritmos IA
        \end{enumerate}
        \item \textbf{Sistema de Reentrada}
        \begin{enumerate}
            \item Escudo térmico
            \item Sistema de Paracaídas
            \item Localización
        \end{enumerate}
        \item \textbf{Integración y Ensayos}
    \end{enumerate}
    \item \textbf{Operaciones}
    \begin{enumerate}
        \item Comunicación Tierra
        \item Monitoreo
        \item Maniobras
    \end{enumerate}
  \item \textbf{Documentación y \acrshort{mcr}}
\end{enumerate}

\chapter{Requerimientos de la Misión}

\begin{figure}[H]
    \centering
    \includegraphics[width=\textwidth]{Imagenes/1. Concept Level/Requirements/Concept_Requirements.png}
\end{figure}
\begin{figure}[H]
    \centering
    \includegraphics[width=\textwidth]{Imagenes/1. Concept Level/Requirements/Environment_Requirements.png}
    \caption{Requerimientos}
    \label{fig:Requerimientos}
\end{figure}

\chapter{Diagrama de Gantt}
\begin{figure}[H]
    \centering
    \includegraphics[scale=0.35]{Imagenes/Diagrama_Gantt_V3.png}
    \caption{Diagrama de Gantt}
    \label{fig:Diagrama de Gantt}
\end{figure}

\newpage
\chapter{RIESGOS}
El proyecto opera en un entorno de alta complejidad, siendo fundamental la gestión activa de riesgos para el éxito de la 
misión, la seguridad del producto final y la integridad de la cápsula de reentrada. Se utiliza una matriz que 
clasifica los riesgos en niveles de \textbf{Bajo}, \textbf{Aceptable}, \textbf{Tolerable} o \textbf{Alto}, ponderando 
Probabilidad e Impacto.

El proyecto opera en un entorno de alta complejidad, abarcando desde el lanzamiento hasta el reingreso y la 
fabricación en condiciones extremas. La identificación y gestión activa de riesgos es fundamental para el éxito de la 
misión y la seguridad del producto final.

Este análisis se centra en evaluar aquellos eventos que podrían comprometer las metas de la misión (principalmente la 
fabricación en \acrshort{leo} y la recuperación de productos ), la integridad de la cápsula de reentrada , o la 
planificación del proyecto (WBS).

Se utiliza una matriz de riesgo que pondera la Probabilidad y el Impacto, permitiendo clasificar los riesgos en 
niveles de Bien, Aceptable, Tolerable o Alto. A continuación, se detallan los riesgos clave, con especial atención a 
las fallas en la separación, la reentrada y la integridad de los componentes críticos:

\begin{itemize}
\item Componentes defectuosos para fabricar la cápsula.
\item Demora/aplazamiento para construir la cápsula.
\item No aprobar los test ambientales de la cápsula.
\item Demora/aplazamiento para lanzar la cápsula, ejemplo fenómenos climáticos
\item La fábrica a instalar en la cápsula no cumple con los estándares de medio ambiente y seguridad para la misión.
\item Falla en el sistema de propulsión de la cápsula: para reposicionar en órbita y hacer reingreso a tierra.
\item Falla en sistema de paracaídas de la cápsula.
\item Falla en la etapa de separación entre el spacecraft y la cápsula.
\item Falla en el módulo de comunicación entre capsula tierra
\end{itemize}


 \textbf{Bajo}, \textbf{Aceptable}, \textbf{Tolerable} o \textbf{Alto}, ponderando Probabilidad e Impacto.

\section{Matriz de Riesgo - Ponderaciones}
\begin{table}[H]
\centering
 \resizebox{\linewidth}{!}{
\begin{tabular}{|l|c|c|c|c|c|}
\hline
\textbf{Probabilidad} & \textbf{Mínimo} (1) & \textbf{Moderado} (2) & \textbf{Serio} (3) & \textbf{Elevado} (4) & \textbf{Grande} (5) \\
\hline
\textbf{Frecuente} (5) & \cellcolor{yellow} 5 & \cellcolor{yellow} 10 & \cellcolor{orange} 15 & \cellcolor{red} 20 (Alto) & \cellcolor{red} 25 (Alto) \\
\textbf{Recurrente} (4) & \cellcolor{green} 4 & \cellcolor{yellow}8 & \cellcolor{orange} 12 (Tolerable) & \cellcolor{red} 16 (Alto) & \cellcolor{red}20 (Alto) \\
\textbf{Posible} (3) & \cellcolor{green} 3 & \cellcolor{yellow} 6 (Aceptable) & \cellcolor{yellow} 9 (Tolerable) & \cellcolor{orange} 12 (Tolerable) & \cellcolor{red} 15 (Alto) \\
\textbf{Inusual} (2) & \cellcolor{green} 2 (Bajo) & \cellcolor{green} 4 (Aceptable) & \cellcolor{yellow} 6 (Aceptable) & \cellcolor{orange} 8 (Tolerable) & \cellcolor{red} 10 (Tolerable) \\
\textbf{Remota} (1) & \cellcolor{green} 1 (Bajo) & \cellcolor{green} 2 (Bajo) & \cellcolor{green} 3 (Bajo) & \cellcolor{yellow} 4 (Aceptable) & \cellcolor{orange} 5 (Aceptable) \\
\hline
\end{tabular}
}
\caption{Clasificación de Nivel de Riesgo}
\end{table}

\section{Riesgos por Categoría}

\begin{longtable}{|p{6cm}|c|c|c|}
\hline
\textbf{Evento} & \textbf{Probabilidad} & \textbf{Impacto} & \textbf{Nivel de Riesgo} \\
\hline
\multicolumn{4}{|l|}{\textbf{General}} \\
\hline
RRHH & Posible & Elevado & \cellcolor{orange} Tolerable \\
Proveedores & Posible & Grande & \cellcolor{red} Alto \\
Componentes & Inusual & Serio & \cellcolor{yellow} Aceptable \\
Fallas SW & Remota & Elevado & \cellcolor{yellow} Aceptable \\
Comunicación & Remota & Serio & \cellcolor{green} Bajo \\
Competidores & Recurrente & Moderado & \cellcolor{yellow} Aceptable \\
Operacional & Posible & Elevado & \cellcolor{orange} Tolerable \\
Climático/Amb & Posible & Elevado & \cellcolor{orange} Tolerable \\
\hline
\multicolumn{4}{|l|}{\textbf{RRHH}} \\
\hline
Falta de personal especializado & Posible & Elevado & \cellcolor{orange} Tolerable \\
Rotación de personal & Recurrente & Moderado & \cellcolor{yellow} Aceptable \\
Sobrecarga laboral & Posible & Moderado & \cellcolor{yellow} Aceptable \\
Problemas de coordinación & Inusual & Moderado & \cellcolor{green} Bajo \\
Falta de capacitación & Recurrente & Elevado & \cellcolor{red} Alto \\
\hline
\multicolumn{4}{|l|}{\textbf{Proveedores}} \\
\hline
Demoras en la entrega & Recurrente & Elevado & \cellcolor{red} Alto \\
Calidad Inconsistente & Posible & Elevado & \cellcolor{orange} Tolerable \\
Dependencia de un único proveedor & Inusual & Moderado & \cellcolor{green} Bajo \\
Riesgo logístico & Recurrente & Serio & \cellcolor{orange} Tolerable \\
Cambios del proveedor & Remota & Elevado & \cellcolor{yellow} Aceptable \\
\hline
\multicolumn{4}{|l|}{\textbf{Componentes}} \\
\hline
Sensores térmicos & Inusual & Grande & \cellcolor{red} Alto \\
Baterías con capacidad insuficiente & Inusual & Elevado & \cellcolor{orange} Tolerable \\
Compatibilidad Imperfecta & Posible & Moderado & \cellcolor{yellow} Aceptable \\
Fallas en el sistema de telemetría & Remota & Elevado & \cellcolor{yellow} Aceptable \\
Degradación térmica del escudo & Inusual & Elevado & \cellcolor{orange} Tolerable \\
\hline
\multicolumn{4}{|l|}{\textbf{Comunicación}} \\
\hline
Pérdida de enlace por visibilidad & Recurrente & Elevado & \cellcolor{red} Alto \\
Interferencia RF & Recurrente & Serio & \cellcolor{orange} Tolerable \\
Pérdida de sincronización temporal & Recurrente & Serio & \cellcolor{orange} Tolerable \\
Fallas en la estación terrena & Inusual & Elevado & \cellcolor{orange} Tolerable \\
Degradación por condiciones & Posible & Moderado &\cellcolor{yellow}  Aceptable \\
\hline
\multicolumn{4}{|l|}{\textbf{Competidores}} \\
\hline
Avances más rápidos de empresas & Recurrente & Serio &  \cellcolor{orange} Tolerable \\
Competidores generan patentes clave & Inusual & Moderado &\cellcolor{green} Bajo \\
Captar a nuestros proveedores críticos & Inusual & Elevado & \cellcolor{orange} Tolerable \\
Competidores con acuerdos gubernamentales & Remota & Moderado & \cellcolor{green} Bajo \\
\hline
\multicolumn{4}{|l|}{\textbf{Operacional}} \\
\hline
Error en la secuencia & Inusual & Elevado & \cellcolor{orange} Tolerable \\
Reentrada mal ejecutada & Posible & Moderado & \cellcolor{yellow} Aceptable \\
Consumo energético mayor & Inusual & Elevado & \cellcolor{orange} Tolerable \\
Colisión con desechos espaciales & Remota & Elevado & \cellcolor{yellow} Aceptable \\
Falla del mecanismo de separación & Posible & Grande & \cellcolor{red} Alto \\
\hline
\multicolumn{4}{|l|}{\textbf{Software}} \\
\hline
Crash del OBC & Posible & Elevado & \cellcolor{orange} Tolerable \\
Fallas en la sincronización de telemetría & Inusual & Serio & \cellcolor{yellow} Aceptable \\
Sobreconsumo de CPU & Inusual & Moderado & \cellcolor{green} Bajo \\
Pérdida o corrupción de datos & Posible & Serio & \cellcolor{yellow} Aceptable \\
Watchdog mal configurado & Inusual & Moderado & \cellcolor{green} Bajo \\
\hline
\multicolumn{4}{|l|}{\textbf{Ambientales}} \\
\hline
Radiación espacial & Recurrente & Serio & \cellcolor{orange} Tolerable \\
Temperaturas extremas & Recurrente & Serio & \cellcolor{orange} Tolerable \\
Interacción con plasma en reentrada & Inusual & Serio & \cellcolor{yellow} Aceptable \\
Impacto de micrometeoritos y debris & Posible & Grande & \cellcolor{red} Alto \\
Contaminación por desgasificación & Remota & Serio & \cellcolor{green} Bajo \\
\hline
\caption{Detalle de la Matriz de Riesgo por Evento}
\end{longtable}

\printbibliography[heading=bibnumbered]

\end{document}
